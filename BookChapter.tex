\documentclass[graybox, envcountchap]{svmult}

\usepackage{mathptmx}        % selects Times Roman as basic font
\usepackage{helvet}          % selects Helvetica as sans-serif font
\usepackage{courier}         % selects Courier as typewriter font
\usepackage{makeidx}         % allows index generation
\usepackage{graphicx}        % standard LaTeX graphics tool
                             % when including figure files
\usepackage{multicol}        % used for the two-column index
\usepackage[bottom]{footmisc}% places footnotes at page bottom
\makeindex             % used for the subject index

\begin{document}

\title{DEM contact models for initial-stage sintering of polymers}
% Use \titlerunning{Short Title} for an abbreviated version
\author{T Weinhart, R Fuchs, M Kappl, S Luding}
% Use \authorrunning{Short Title} for an abbreviated version
\institute{
Thomas Weinhart, Stefan Luding \at University of Twente, \email{t.weinhart@utwente.nl, s.luding@utwente.nl} \and 
Regina Fuchs \at Evonik Technology \& Infrastructure, \email{regina.fuchs@evonik.com} \and 
Michael Kappl \at MPI for Polymer Research, \email{kappl@mpip-mainz.mpg.de}}
\maketitle

\abstract{Abstract}

\section{Introduction}
Modelling of particle systems: DEM allows study of complex bulk behaviour (force chains, fracture, etc). [Take from Proposal \S1.5]

\section{Contact modelling friction}
Review contact models: Elasto-viscous-frictional behaviour

\section{Contact models for sintering -- pressure- and temperature- dependency}
Stefan's sintering, other sintering models, our new sintering model.

How to incorporate this into a sintering model. Show details. How to scale if Collision << Sintering time scale. 


\subsection{Experimental validation friction}
Nanoindenter studies.

\section{Experiments sintering}
Indentation tests

\section{Results}
Look at what we learn from indentation tests: We see different behaviour for r<r_crit.

\section{Conclusion}

\section{Outlook}






Section 1 Viscous Sintering: 
Review idea of viscous sintering near glass temperature. Frenkel's predictions

Section 2 Experimental Observations: 


Section 3 Modelling: 

Section 4 Validation:
Reproduce results of indentation tests.

\section{Section Heading}
\label{sec:1}
Section

\bibliographystyle{spbasic}
\bibliography{papers}

\end{document}

